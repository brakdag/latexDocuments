\documentclass[a4paper]{article}
\usepackage[utf8]{inputenc}
\usepackage[spanish]{babel}
\usepackage{fancyhdr}
\usepackage{vmargin}

\setpapersize{A4}
\setmargins{2.5cm}       % margen izquierdo
{1.5cm}  % margen superior
{16.5cm} % anchura del texto
{23.42cm}  % altura del texto
{14pt}  % altura de los encabezados
{1cm}  % espacio entre el texto y los encabezados
{0pt} % altura del pie de página
{1cm}   % espacio entre el texto y el pie de página
\title{Trabajo Práctico 2- Instalaciones Eléctricas}
\author{Gustavo David Ferreyra }
\date{Marzo 2018}

\begin{document}

\fancyhead[L]{\textit{INSTALACIONES ELÉCTRICAS.}}
\fancyhead[C]{}
\fancyhead[R]{Potencia - Trabajo Práctico 2}
\fancyfoot[L]{\rule{167mm}{0.1mm}                 Escuela 4-117 'Ejército de los Andes' gustavoferreyra@outlook.com}
\fancyfoot[C]{}
\fancyfoot[R]{}
\pagestyle{fancy} 
\section{Cálculo de Potencia}
\subsection{Potencia}
\subsubsection{Resolver: Justificar en cada caso su respuesta}
\begin{enumerate}
\large    
\item Si la corriente que circula por una resistencia es de  $1A$ y su resistencia es de $10\Omega$. Calcular la potencia disipada en la resistencia.

\item La potencia disipada en una resistencia conectada a 10V es de 100W, Calcular su resistencia.
\item Un artefacto resistivo puro de $12\Omega$ se conecta a 12V. Calcular la potencia que consume el artefacto.
\item Una resistencia calefactora de 1CV se conecta a 220V, calcular su resistencia.
\item La potencia de una horno eléctrico resistivo es de 2HP, si se conecta a 220V calcular la corriente que consume.
\item El filamento de una lámpara incandescente tiene un consumo de $100W$ conectado a $220V$, se corta el filamento a la mitad y se conecta nuevamente,¿Cuánta potencia va a consumir?
\item Se tiene una estufa eléctrica de 1000W de consumo y 110V, si puede conectar también a 220V,¿Cuál será su consumo?
\item La potencia disipada por una lámpara incandescente conectada a 110V es de 50W, si esta se conecta a 220V ¿En cuánto se incrementa la corriente?
\item Se conectan 2 lámparas de 100w en paralelo, si las conecto a 220v ¿Cuál será el consumo?.
\item Si las lámparas del ejercicio anterior las conecto en serie. ¿Cuál será el consumo?
\item Se conectan 3 resistencias en paralelo, por cada una de ellas ciruclan 3A,y están conectadas a 220V. Calcular la potencia del conjuto. 
\item Se conectan las 3 resistencias del problema anterior en serie. Calcular la potencia del conjunto.
\item Se desea conectar una lámpara de 110V a 220V para ello se piensa conectar en serie una resistencia limitadora, calcular el valor de la resistencia y la potencia que debe disipar si se sabe que la lámpara tiene una potencia de 5W.
\item Se desea conectar un led que consumo 15mA a 220V, para ello se necesita conectar una resistencia limitadora, calcular el valor de la resistencia y su potencia.
\item Se desea utilizar un generador de 250V para iluminar un escenario de un espectaculo que tiene un consumo de 1000W, si los cables son de cobre y de $2mm^2$. Calcular a que distancia máxima se puede colocar. La tensión en las lámparas no debe ser menor a 220V.
\item Dadas dos resistencias de $3\Omega$ y $400\Omega$ conectadas en paralelo a 12V, calcular la potencia consumida por en cada una.
\item Calcular la potencia en HP consumida por las resistencias del problema anterior conectadas en serie a 12V.
\item 
\end{enumerate}

\end{document}
