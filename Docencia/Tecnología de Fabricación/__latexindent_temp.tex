 % !TEX program = xelatex
\documentclass[a4paper]{article}
\usepackage[utf8]{inputenc}
\usepackage[spanish]{babel}
\usepackage{fancyhdr}
\usepackage{vmargin}
\usepackage{graphicx}

\setpapersize{A4}
\setmargins{2cm}       % margen izquierdo
{1.5cm}  % margen superior
{17cm}%{16.5cm} % anchura del texto
{23.42cm}  % altura del texto
{13.2pt}  % altura de los encabezados
{1cm}  % espacio entre el texto y los encabezados
{0pt} % altura del pie de página
{1cm}   % espacio entre el texto y el pie de página
\title{Tecnología de Fabricación}
\author{Gustavo David Ferreyra }
\date{Marzo 2018}

\begin{document}

\fancyhead[L]{\textit{TECNOLOGÍA DE FABRICACIÓN.}}
\fancyhead[C]{}
\fancyhead[R]{Control Numérico - Trabajo Evaluativo}
\fancyfoot[L]{\rule{167mm}{0.1mm}                 Escuela 4-117 'Ejército de los Andes' gustavoferreyra@outlook.com}
\fancyfoot[C]{}
\fancyfoot[R]{}
\pagestyle{fancy} 
\section{Herramientas de Mecanizado CN}
%\begin{center}
%\includegraphics[width=50mm]{engranajellave.png}
%\end{center}
\subsection{Herramientas de corte}

\begin{enumerate}
    \item Explique que es una \textit{herramienta de corte}.
   \item Propiedades del tipo de material en \textit{herramientas de corte}.
   \begin{itemize}
    \item Acero rápido.
    \item Metal duro.
    \item Materiales cerámicos: \textit{oxido de aluminio,nitruro/carburo de silicio}.
    \item Nitruro de boro cúbico.
    \item Diamante policristalino.
  \end{itemize}
   \item ¿Cuáles son las principales características y partes de las \textit{herramientas para torno}?
    \item Explique los tipos de recubrimientos utilizados en \textit{herramientas de corte}.
    \item ¿Qué es un \textit{rompevirutas}? Clasifique los \textit{rompevirutas} según su uso.
    \item Explicar los diferentes mecanismos de desgaste de las \textit{herramientas de corte}.
    \begin{itemize}
      \item Abrasión.
      \item Difusión.
      \item Oxidación.
      \item Adhesión.
    \end{itemize}
    \item Explique la nomenclatura de herramientas según la \textit{norma ISO-1832}.
    \item Clasifique los \textit{insertos} para herramientas de torno.
    \item Clasifique los diferentes tipos de herramientas para torno.
    \item ¿Qué es un \textit{porta inserto} de herramienta exterior e interior?
    \item Listar y explicar herramientas para fresadora.
    \item Explicar usos, dibujar, material y forma geométrica de:
    \begin{itemize}
      \item Brocas normales helicoidales.
      \item Brocas para perforar hormigón.
      \item Brocas para perforar piezas cerámicas y vidrio.
      \item Broca larga.
      \item Broca super larga.
      \item Brocas de centrar.
      \item Broca para berbiquí
      \item Broca de paleta.
      \item Broca para excavación o Trépano.
      \item Brocas para máquinas de control numérico.
    \end{itemize} 
    \item En caso de que durante el mecanizado de una pieza, se observe una rotura de la herramienta.
    ¿Qué parámetros de mecanizado se podrían modificar? ¿Cómo afectarían a la productividad?
     \item Listar bibliografía consultada utilizando las normas APA.
  \end{enumerate}


\end{document}
