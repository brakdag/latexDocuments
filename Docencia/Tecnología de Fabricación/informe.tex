 % !TEX program = xelatex
\documentclass[a4paper]{article}
\usepackage[utf8]{inputenc}
\usepackage[spanish]{babel}
\usepackage{fancyhdr}
\usepackage{vmargin}
\usepackage{graphicx}

\setpapersize{A4}
\setmargins{2cm}       % margen izquierdo
{1.5cm}  % margen superior
{17cm}%{16.5cm} % anchura del texto
{23.42cm}  % altura del texto
{13.2pt}  % altura de los encabezados
{1cm}  % espacio entre el texto y los encabezados
{0pt} % altura del pie de página
{1cm}   % espacio entre el texto y el pie de página
\title{Tecnología de Fabricación}
\author{Gustavo David Ferreyra }
\date{Marzo 2018}

\begin{document}

\fancyhead[L]{\textit{TECNOLOGÍA DE FABRICACIÓN.}}
\fancyhead[C]{}
\fancyhead[R]{Control Numérico - Trabajo Evaluativo}
\fancyfoot[L]{\rule{167mm}{0.1mm}                 Escuela 4-117 'Ejército de los Andes' gustavoferreyra@outlook.com}
\fancyfoot[C]{}
\fancyfoot[R]{}
\pagestyle{fancy} 
\section{Herramientas de corte.}
%\begin{center}
%\includegraphics[width=50mm]{engranajellave.png}
%\end{center}
\subsection{Informe Evaluativo}

\begin{enumerate}
    \item Explique que es una herramienta de corte y cuales son las características principales que debe poseer.
    \item Explique de que tipo de materiales son fabricadas las herramientas de corte, y que propiedades le da cada material.
    \begin{itemize}
      \item Acero rápido.
      \item Metal duro.
      \item materiales cerámicos.
      \item Nitruro de boro cúbico.
      \item Diamante policristalino.
    \end{itemize}
    \item ¿Qué es un rompevirutas? Clasifique los rompevirutas según su uso.
    \item Explique los tipos de recubrimientos utilizados en herramientas de corte.
    \item Explicar los diferentes mecanismos de desgaste de las herramientas de corte.
    \begin{itemize}
      \item abrasión.
      \item difusión.
      \item oxidación.
      \item adhesión.
    \end{itemize}
    \item Explique la nomenclatura de herramientas según la \textit{norma ISO-1832}.
    \item Clasifique los insertos para herramientas de torno tipo trigon WNMG/A y TNMG.
    \item Clasifique los diferentes tipos de herramientas para torno.
    \item ¿Qué es un porta inserto de herramienta exterior e interior?
    \item Listar y explicar herramientas para fresadora.
    \item Explicar usos, materiales con que están construídas y forma de:
    \begin{itemize}
      \item Brocas normales helicoidales.
      \item Brocas para perforar hormigón.
      \item Brocas para perforar piezas cerámicas y vidrio.
      \item Broca larga.
      \item Broca super larga.
      \item Brocas de centrar.
      \item Broca para berbiquí
      \item Broca de paleta.
      \item Broca para excavación o Trépano.
      \item Brocas para máquinas de control numérico.
    \end{itemize} 
\end{enumerate}

\end{document}
