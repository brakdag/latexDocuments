\documentclass[14pt,a4paper]{article}
\usepackage[utf8]{inputenc}
\usepackage[spanish]{babel}
\usepackage{fancyhdr}
\usepackage{vmargin}

\setpapersize{A4}
\setmargins{2.5cm}       % margen izquierdo
{1.5cm}  % margen superior
{16.5cm} % anchura del texto
{23.42cm}  % altura del texto
{12pt}  % altura de los encabezados
{1cm}  % espacio entre el texto y los encabezados
{0pt} % altura del pie de página
{1cm}   % espacio entre el texto y el pie de página
\title{Trabajo Práctico 1- Instalaciones Eléctricas}
\author{Gustavo David Ferreyra }
\date{Marzo 2018}

\begin{document}

\fancyhead[L]{\textit{INSTALACIONES ELÉCTRICAS.}}
\fancyhead[C]{}
\fancyhead[R]{Conductores y cálculo de resistividad - Trabajo Práctico 1}
\fancyfoot[L]{\rule{167mm}{0.1mm}                 Escuela 4-117 'Ejército de los Andes' gustavoferreyra@outlook.com}
\fancyfoot[C]{}
\fancyfoot[R]{}
\pagestyle{fancy} 
\section{Conductores}
\subsection{Cálculo de resistividad}
\subsubsection{Resolver: Justificar en cada caso su respuesta}
\begin{enumerate}
\large    
    \item Un conductor de cobre tiene una resistencia de 1$\Omega$. Si se triplica la longitud. ¿Cuál será el valor de su resistencia en Ohms?
    \item El conductor del problema anterior tiene sección cuadrada, si se duplica su sección. ¿Cuál será el valor de su resistencia?
    \item Si la longitud inicial del conductor del problema 1 es de $20m$ de largo. ¿Qué longitud debe tener el conductor para que disminuya la resistencia a $0.4\Omega$?
    \item Un conductor de sección circular y 40 metros de longitud, tiene un diámetro de 2mm. Otro conductor mide 30 metros de largo y tiene un diámetro de 1mm. Si en ambos se mide el mismo valor de resistencia $4\Omega$, ¿Están hechos del mismo material?   
    \item Calcular la resistividad para un conductor que posee $2\Omega$, tiene un diámetro de 3mm y una longitud de 10 metros.
    \item Calcular la resistividad para un conductor que posee $10\Omega$, tiene un radio de 3mm y una longitud de 40 metros.
    \item Un cable metálico parece ser buen conductor, sobre una longitud 5m y 1mm de diámetro se midió $0.2\Omega$, ¿Qué resistencia tendrá un cable fabricado con el mismo material de 40m largo y 2mm radio?
    \item Se conecta un conductor a una batería de $9V$ y se mide con un amperímetro que el cortocircuito marca $4A$. Si el conductor se corta a la mitad, y se vuelve a conectar. ¿Cuánto debería marcar el amperímetro?
    \item ¿Cuál es la resistividad del conductor del problema anterior si tiene una longitud de 2m y una sección de $2mm^2$?
    \item Un conductor de cobre tiene una resistividad $0.0171 \Omega mm^2/m$. Si tiene una longitud de 100m y una sección de $4mm^2$. Calcular su resistencia.
    \item Calcular el valor de la resistencia en Ohms($\Omega$) para un conductor de cobre con resistividad igual a la del problema anterior, que tiene una longitud de 1000m y un diámetro de $5mm$.
    \item La resistividad para el cobre tiene dos representaciones $0.0171 \Omega \frac{mm^2}{m}$ y $1,71 x 10^{-8}\Omega m$ Si el hierro tiene una resistividad de $8,90 x 10^{-8}\Omega m$ ¿Cual será su valor en $\Omega \frac{mm^2}{m}$ ?
    \item Elaborar una tabla con resistividades en $\Omega \frac{mm^2}{m}$ y $\Omega m$ para los materiales cobre,hierro,plata, oro,estaño, platino, aluminio y grafito.
    \item El conductor del problema 1.¿Es de realmente de cobre?
\end{enumerate}

Siendo:
$\rho$:Resistividad.
$s:$Sección del conductor.
$l:$longitud del conductor.

\centering
\fbox{
\Huge

$R=\rho (\frac{l}{s}) \Rightarrow \rho = R (\frac{s}{l}) $
}


\end{document}
