 % !TEX program = xelatex
\documentclass[a4paper]{article}
\usepackage[utf8]{inputenc}
\usepackage[spanish]{babel}
\usepackage{fancyhdr}
\usepackage{vmargin}
\usepackage{graphicx}

\setpapersize{A4}
\setmargins{2cm}       % margen izquierdo
{1.5cm}  % margen superior
{17cm}%{16.5cm} % anchura del texto
{23.42cm}  % altura del texto
{13.2pt}  % altura de los encabezados
{1cm}  % espacio entre el texto y los encabezados
{0pt} % altura del pie de página
{1cm}   % espacio entre el texto y el pie de página
\title{Trabajo Práctico 2- Instalaciones Eléctricas}
\author{Gustavo David Ferreyra }
\date{Marzo 2018}

\begin{document}

\fancyhead[L]{\textit{MANTENIMIENTO Y REPARACIÓN DE EQUIPOS.}}
\fancyhead[C]{}
\fancyhead[R]{Mantenimiento Predictivo - Trabajo Evaluativo}
\fancyfoot[L]{\rule{167mm}{0.1mm}                 Escuela 4-117 'Ejército de los Andes' gustavoferreyra@outlook.com}
\fancyfoot[C]{}
\fancyfoot[R]{}
\pagestyle{fancy} 
\section{Mantenimiento Predictivo}
\begin{center}
\includegraphics[width=50mm]{engranajellave.png}
\end{center}
\subsection{Análisis de fallos}

\begin{enumerate}
    
    \item Realizar una \textit{ficha de equipo} para 2 (dos) máquinas. Incluir explicación del entorno de trabajo y \textit{análisis de criticidad} (tabla de criticidad) selección de \textit{modelo de mantenimiento} (En diagrama de flujo). 
    \item Armar una tabla con los posibles \textit{fallos funcionales, y técnicos} de la máquina. Mínimo 10 (diéz).

  \begin{table}[htbp]
\begin{center}
\begin{tabular}{|l|l|}
\hline
Fallo Funcional & Fallo Técnico \\
\hline \hline
Corte de cadena & Ruido cadena \\ \hline
Corte de eje & Ruido de frenos \\ \hline
Corte de frenos & Daño en pintura \\ \hline
... & ... \\ \hline
\end{tabular}
\caption{Ejemplo de tabla de fallos.}
\label{tabla:sencilla}
\end{center}
\end{table}

    \item Realizar una lista de los datos a ser monitoreados y almacenados  para el \textit{mantenimiento predictivo.}
    \item Realizar un \textit{diagrama de pez}  causa-efecto hombre /máquina/entorno/ material/método/medida. para el equipo elegido.
   \item Realizar una investigación de las causas de 3 (Tres) fallas. Utilizando la técnica de las 5 preguntas : ¿Quién o quienes?,¿Por qué?,¿Dónde?,¿Cuál?,¿Qué?.
    \item Analizar las fallas con respecto a materiales de alguna de los componentes de la máquina, teniendo en cuenta para el análisis de falla del material, composición del material, precedentes de casos similares, medio ambiente, condiciones de operación, estados de esfuerzos y factores humanos,
\item Identificiación y análisis de un problema.
Definir el problema y reconocer su importancia.
  \begin{table}[htbp]
\begin{center}
\begin{tabular}{|l|l|}
\hline
Acción & Descrpición  \\
\hline \hline
Problema & Explicación detallada del fallo \\ \hline
Histórico del problema & Datos historicos acerca de cada cuanto tiempo puede producirse el fallo o diversos\\
 & indicadores que deben tenerse en cuenta para poder predecir un fallo. \\ \hline
Pérdidas actuales & Daño económico que produce la rotura o fallo por disminución de producción\\
& o pérdida de calidad \\ \hline
Análisis de Pareto & Donde estaría ubicado dentro de los \\
& fallos posibles como un fallo 80-20 \\ \hline
\end{tabular}
\caption{Ejemplo identificación de problemas.}
\label{tabla:sencilla}
\end{center}
\end{table}

\item Al Listar referencias y bibliografía utilizar normas APA.

\end{enumerate}
\end{document}
