 % !TEX program = xelatex
\documentclass[a4paper]{article}
\usepackage[utf8]{inputenc}
\usepackage[spanish]{babel}
\usepackage{fancyhdr}
\usepackage{vmargin}

\setpapersize{A4}
\setmargins{2.5cm}       % margen izquierdo
{1.5cm}  % margen superior
{17cm}%{16.5cm} % anchura del texto
{23.42cm}  % altura del texto
{13.2pt}  % altura de los encabezados
{1cm}  % espacio entre el texto y los encabezados
{0pt} % altura del pie de página
{1cm}   % espacio entre el texto y el pie de página
\title{Trabajo Práctico 1- Instalaciones Eléctricas}
\author{Gustavo David Ferreyra }
\date{Marzo 2018}

\begin{document}

\fancyhead[L]{\textit{MANTENIMIENTO Y REPARACIÓN DE EQUIPOS.}}
\fancyhead[C]{}
\fancyhead[R]{Lista de Equipos - Trabajo Práctico 1}
\fancyfoot[L]{\rule{167mm}{0.1mm}                 Escuela 4-117 'Ejército de los Andes' gustavoferreyra@outlook.com}
\fancyfoot[C]{}
\fancyfoot[R]{}
\pagestyle{fancy} 
\section{Lista de Equipos}
\subsection{Codificación de equipos}
\subsubsection{Actividad 1: Leer atentamente y el siguiente proceso de fabricación, elaborar la lista de equipos:Proceso de fabricación de cuadernos Austral}
\begin{enumerate}
\large
\item Los rollos de papel entran en la máquina

El principal insumo para la fabricación de cuadernos es el papel. Es por ello que en Austral existe una bodega llena de grandes rollos de papel, llamados Bobinas, que pesan un promedio de 800 kilos c/u y se van apilando unos sobre otros formando altas columnas, a la espera de entrar en la gran máquina que los transformará en cuadernos.

El proceso de fabricación comienza con la colocación de estos rollos en la máquina, donde el papel transita a una cierta velocidad por una serie de cilindros que cumplen la función de dejarlo perfectamente estirado y sin ondas.

Además, nuestra máquina nos permite poner en ella 2 rollos de papel simultáneamente, posibilitando que el proceso nunca se detenga, ya que al terminarse el papel del primer rollo, inmediatamente se empieza a usar el del segundo.

\item Rayado de hojas y corte en pliegos

Una vez alisado el papel, ingresa a la etapa de impresión de los distintos rayados que puede tener un cuaderno (matemáticas 7mm, matemáticas 5mm, composición, caligrafía horizontal, caligrafía vertical y ciencias).

Básicamente, esta fase consiste en hacer pasar el papel por unos cilindros especiales que tienen marcado en relieve el dibujo del rayado y que se ha impregnado de tinta.

Una vez terminada la impresión del rayado se procede al corte del rollo en pliegos de hojas.

\item Colocación de tapas y contratapas

Los pliegos de hojas son apilados en grupos, según el número de hojas especificadas para el cuaderno en fabricación. Luego, y obedeciendo a la indicación de un contador de hojas, la máquina procede a colocar un pliego impreso de tapas y contratapas a los grupos de hojas en pliegos.

\item Apilado en pliegos y trazado a tamaño individual

Al ingresar a esta fase, los cuadernos aún están unidos en un mismo pliego. Esta etapa, entonces, consiste en trozar estos pliegos para formatear los cuadernos a su tamaño individual. Este proceso se realiza en forma continua con guillotinas especiales de gran precisión.

\item Perforado y espiralado

Ya en su formato individual, los cuadernos avanzan, uno tras otro, por una correa transportadora hacia la etapa de perforado. Ésta consiste en perforar todo el lateral izquierdo del cuaderno con pequeños orificios donde enseguida se le pondrá el espiral -doble o simple, según corresponda.

\item Apilado

El cuaderno ya está listo, pero continúa transitando por una correa transportadora que lo lleva a la etapa siguiente: el proceso de apilado. Aquí, los cuadernos se apilan uno sobre otro en grupos de diez, o según la medida de empaque que se quiera. Los cuadernos seguirán avanzando en el proceso, pero ahora van en grupos.

\item Empaquetado y sellado

Los grupos de cuadernos se introducen en una bolsa plástica; este paquete es introducido en un horno donde se le aplica calor para el sellado del empaque. En esta etapa se identifica a los paquetes con una etiqueta que contiene información sobre el tipo de cuaderno, la cantidad por paquete, y el diseño y la línea de que se trate.

\item Encajado y entramado

Una vez que los cuadernos están empaquetados, son ubicados al interior de cajas etiquetadas con la información del tipo de producto que contienen; a su vez, estas cajas son colocadas sobre tarimas para su almacenamiento en las bodegas de productos terminados.

Finalizado el proceso de fabricación, los cuadernos Austral quedan listos para ser puestos a disposición de estudiantes y usuarios en todo el país.

\end{enumerate}
\large
\subsubsection{Actividad 2: Responder el las siguientes preguntas.}
\begin{enumerate}
\large
    \item ¿Cuántas plantas son necesarias en este proceso de fabricación? \item ¿Podría dividirse cada planta en áreas?
    \item ¿Cuántos equipos hay en cada planta/área?
    \item ¿Qué proceso podría estar automatizado y cuál no?
    \item ¿Qué insumos están presentes y que diferencia existe entre insumo y equipo?
    \item ¿Qué proceso necesitaría de control a través de papeles (documentos por escrito)?
    \item ¿Qué equipos que el texto no menciona debe poseer Austral?
    \item ¿Qué equipos son susceptibles de romperse?
    \item El sector de mantenimiento desea elaborar una lista de equipos. Elabore una lista de equipos.
    \item ¿Cómo identificaría cada equipo, para reconocer que es el de su lista?
    \item ¿El gerente de la empresa quiere conocer una estimación del costo anual de mantenimiento?¿Qué estrategias podría usar para determinarlo?
    \item Elaborar una lista de equipos estimada para los siguientes procesos de fabricación:
    \begin{enumerate}
        \item Fabricación de sillas.
        \item Fabricación de martillos.
        \item Fabricación de percheros. 
        \item Ensamblado de bicicletas.
        \item Ensamblado de notebooks.
    \end{enumerate}
    
\end{enumerate}

\end{document}
