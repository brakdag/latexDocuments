 % !TEX program = xelatex
\documentclass[a4paper]{article}
\usepackage[utf8]{inputenc}
\usepackage[spanish]{babel}
\usepackage{fancyhdr}
\usepackage{vmargin}

\setpapersize{A4}
\setmargins{2cm}       % margen izquierdo
{1.5cm}  % margen superior
{17cm}%{16.5cm} % anchura del texto
{23.42cm}  % altura del texto
{13.2pt}  % altura de los encabezados
{1cm}  % espacio entre el texto y los encabezados
{0pt} % altura del pie de página
{1cm}   % espacio entre el texto y el pie de página
\title{Trabajo Práctico 2- Instalaciones Eléctricas}
\author{Gustavo David Ferreyra }
\date{Marzo 2018}

\begin{document}

\fancyhead[L]{\textit{MANTENIMIENTO Y REPARACIÓN DE EQUIPOS.}}
\fancyhead[C]{}
\fancyhead[R]{Mantenimiento correctivo - Trabajo Práctico 2}
\fancyfoot[L]{\rule{167mm}{0.1mm}                 Escuela 4-117 'Ejército de los Andes' gustavoferreyra@outlook.com}
\fancyfoot[C]{}
\fancyfoot[R]{}
\pagestyle{fancy} 
\section{Tipos de Mantenimiento}
\subsection{Mantenimiento Correctivo}
\subsubsection{Actividad 1:Resolver los siguientes problemas. Cantidades de producción y mantenimiento correctivo}
\begin{enumerate}
    

    \item Un torno paralelo sufre roturas funcionales después de 10 años de uso, que lo dejan sin funcionar, si se realiza un mantenimiento correctivo el costo es de 10000\$ a diferencia
    de un preventivo diario con un costo de 30\$. ¿Qué tipo de mantenimiento es más económico?
    \item Un filtro purificador de agua requiere mantenimiento correctivo, 
    100 litros de agua comienza a filtrar y aumenta el 0,01\% de impurezas cada 10 litros.
    Si se necesita que el agua tenga un 97\% de pureza ¿Cada cuantos litros hay que cambiar el filtro?
    \item Las impurezas en una aceite lubricante para un automovil aumentan con el paso del tiempo siguiendo esta curva $y=t^2/9$ siendo t el tiempo transcurrido en meses, y y   
    \item Una máquina que fabrica cajas la cual no recibe mantenimiento, reduce la calidad de su producción 
    en función de la cantidad producida. Cada 23 unidades producidas, produce 1 caja 
    defectuosas más. ¿Cuál será la cantidad máxima producida? La máquina deja de producir cuando 
    todas las unidades que produce son defectuosas.

    
\begin{table}
    \begin{center}
    \begin{tabular}{|c| c | c | }
        \hline       
          Producción & bien & defectuoso  \\   
        \hline  
          23 & 23 & 0 \\
          23 & 22 & 1 \\
          23 & 21 & 2 \\
          23 & 20 & 3 \\
        \hline  
        \hline
        \hline       
        Producción & bien & defectuoso  \\          
          92 & 86 & 6 \\
    \hline  
    \end{tabular}
\end{center}
    \label{Ejemplo para 92 unidades producidas}
    \caption{Ejemplo para 92 unidades producidas(Problema 4)}
\end{table}
    \item En problema anterior si se realiza un mantenimiento correctivo cada 30 cajas, cuantas cajas defectuosas habrá
    si se producen 6000 cajas.
    \item Con respecto al problema anterior. Si el costo de producir 1 caja es de 10\$ y el mantenimiento correctivo tiene un costo de 100\$ ¿Conviene económicamente realizar dicho mantenimiento? 

\end{enumerate}

\subsubsection{Actividad 2: Resolver los siguientes problemas. Nuevo o usado mantenimiento correctivo}
\begin{enumerate}
\item Una de las decisiones más importante del mantenimiento correctivo,
 es decidir acerca de reemplazar el equipo por uno nuevo o realizar 
 reparaciones por fallos funcionales frecuentes a través de mantenimiento 
 correctivo. Si una máquina que trabaja 24hs diarias cada 10 horas tiene
un fallo funcional, que requiere 1 hora de repearación, por otra parte,
una máquina nueva funciona sólamente 18 hs al día pero se rompe cada 3hs
 y su reparación tarda 15 minutos. ¿Cuál máquina funciona más tiempo durante un año de uso?
\item El mantenimiento correctivo de un taladro de pie antiguo en un una planta metalmecánica 
se realiza semestralmente con un costo de 1000\$, si se desea reemplazar el taladro por uno nuevo 
cuyo costo de mantenimiento es de 100\$ mensual, ¿A partir de cuantos años de trabajo es rentable hacer el cambio? 
\item Una máquina nueva 200000\$ tiene fallos funcionales cada 1000 unidades con un costo de 100\$ de reparación. 
Por otra parte una máquina usada 100000\$ tiene costo de mantenimiento 50\$ cada 300 unidades.
Si se van a producir solo 20000 unidades. ¿Cuál máquina es conveniente elegir?
Si se van a producir solo 100 unidades. ¿Cuál máquina es conveniente elegir?
\item 

\end{enumerate}
\end{document}
