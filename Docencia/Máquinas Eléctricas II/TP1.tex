\documentclass[a4paper]{article}
\usepackage[utf8]{inputenc}
\usepackage[spanish]{babel}
\usepackage{fancyhdr}
\usepackage{vmargin}
\setpapersize{A4}
\setmargins{2.5cm}       % margen izquierdo
{1.5cm}  % margen superior
{16.5cm} % anchura del texto
{23.42cm}  % altura del texto
{14pt}  % altura de los encabezados
{1cm}  % espacio entre el texto y los encabezados
{0pt} % altura del pie de página
{1cm}   % espacio entre el texto y el pie de página
\title{Trabajo Práctico 1- Máquinas Eléctricas II}
\author{Gustavo David Ferreyra }
\date{Marzo 2018}

\begin{document}

\fancyhead[L]{\textit{MÁQUINAS ELÉCTRICAS II.}}
\fancyhead[C]{}
\fancyhead[R]{Electromagnetismo - Trabajo Práctico 1}
\fancyfoot[L]{\rule{167mm}{0.1mm}                 Escuela 4-124 'Reynaldo Merín' gustavoferreyra@outlook.com}
\fancyfoot[C]{}
\fancyfoot[R]{}
\pagestyle{fancy} 
\section{Electromagnetismo - Trabajo Práctico 1}
\subsection{Campo magnético creado por una corriente eléctrica, intensidad de campo, inducción,permeabilidad magnética,fuerza magnetomotriz,electroimanes}
\subsubsection{Resolver los siguientes problemas:}

\begin{table}
\caption{Relación entre intensidad de campo magnético $H$ e Inducción magnética $B$.\label{long}}
\begin{center}
\begin{tabular}[c]{|p{2cm}|p{3cm}|p{3cm}|p{3cm}|}

 \hline
  & \multicolumn{3}{|c|}{$H(Av/m)$} \\
 \hline
 {$B(Teslas)$} & Hierro forjado & Chapa normal & Chapa Silicio \\
 \hline
 0,1 & 80 & 50 & 90 \\
 0,3 & 120 & 65&140 \\
 0,5&160&100&170\\
 0,7&230&180&240\\
 0,9&400&360&350\\
 1,1&650&675&530\\
 1,3&1000&1200&1300\\
 1,5&2400&2200&500\\
 1,6&5800&3500&9000\\
 1,7&7000&5000&16500\\
 1,8&11000&10000&27500\\
 1,9&17000&16000&\\
 2&27000&32000&\\
 \hline
\end{tabular}
\end{center}
\end{table}

\begin{enumerate}
\large
    \item ¿Cuál es la inducción magnética $B$ existente en la cara de un imán de sección circular del parlante de un equipo de audio con radio $r=3cm$ si es atravesado por un flujo de $\Phi = 8mWb$?
    \item ¿Cuál es el flujo magnético $\Phi$ que existe en el campo magnético producido por una bobina de una electroválvula si tiene esta un núcleo cilíndrico de diámetro $d=4cm$ y la inducción magnética en la misma es de $B=1,5T$?
    \item ¿Cuál sería la sección de un electroimán que teniendo una inducción de $B=1T$ produce un flujo magnético de $\Phi=10mWb$?
    \item Se desea fabricar un electroimán de fuerza magnetomotriz $\mathcal{F}=221Av$, si solamente se puede fabricar con un número entero de vueltas y corriente (ej:1v,2v,3v...nv)(ej:1A, 2A, 3A ...nA) ¿Cuáles son las $4$ únicas formas de fabricación posibles?
    \item Calcular la intensidad del campo $H$ en el interior de una bobina que tiene 400 vueltas, una corriente $I=9A$ y una longitud media de $L=60cm$.
    \item Si el campo magnético que se produce en una bobina de $L=80cm$ al circular $I=2A$ de corriente por ella es de $H=5000Av/m$, calcular el número de espiras que posee.
    \item Una intensidad de campo de $H=4000Av/m$ produce una $f.m.m.$ de $500Av$. Averiguar la longitud media del núcleo.
    \item Calcular el campo magnético $H$ de un electroimán solenoide comercial de $24VDC$ que también funciona con $12VDC$. Si este en $24VDC$ funciona con $I=0.43A$ tiene una bobina de 1000 vueltas y 38mm de longitud de núcleo¿Tendrá la misma intensidad de campo con ambas tensiones?
    \item Comercialmente se venden solenoides que son electroimanes de núcleo móvil, que tienen $5Kg$ de fuerza de agarre, y funcionan con $220V$. La fuerza que puede realizar está dada por la siguiente fórmula $ F = 40000 B^2 S$. Sabiendo que la sección es cuadrada de 2cm de lado. Determinar la inducción magnética $B$ necesaria para que dicha fuerza sea posible.
    \item Hay electroimanes en lavadoras automáticas para accionar el cierre de válvulas de purga, estas requieren de fuerza de apertura como para cierre, el cierre es producido por un resorte antagonista conectado a un núcleo móvil que cuando se energiza vence la fuerza del resorte e impulsa al núcleo hacia dentro del solenoide solidario a la válvula. Si la electroválvula posee un núcleo de acero forjado de longitud de núcleo $L=2cm$ y sección $s=2cm^2$ , una bobina de 200 vueltas y consume una corriente $I=0,58A$ ¿Cuál es la fuerza de accionamiento?
    \item Se está fabricando un brazo robot que necesita levantar una lata de conservas de durazno al natural que está hecha de hojalata recubierta por una capa de estaño despreciable (equivalente a chapa normal), para esto el equipo de producción pensó en diferentes alternativas, una de ellas es fabricar un electroimán. para ello se hizo un electroimán con núcleo de chapa silicio con las siguientes características: $N=150$, $i=0.4A$,$S=3cm^2$,$L=3cm$ ¿Podrá levantar la lata?
    \item Determinar la fuerza con la que atraerá un electroimán de armadura de hierro si la inducción que aparece en el núcleo es de $1,3T$ y la superficie de total de contacto entre el núcleo y el hierro móvil es de $4cm^2$ 
    \item Se desea conseguir que el electroimán de un contacto automático desarrolle una fuerza de atracción al bloque de contactos móviles de $2Kg$. Teniendo en cuenta que el núcleo está fabricado de hierro forjado, que posee 2000 espiras y que las dimensiones y forma del circuito magnético de dicho electroimán son las que se muestran en la siguiente figura, calcular la intensidad de la corriente eléctrica para conseguirlo. 
    \item Calcular la permeabilidad absoluta para una lámina de hierro que si es afectada por una intensidad de campo de 400Av/m adquiere una inducción magnética $B=1Tesla$.
    \item Si  $\mu_r=\mu/\mu_o$ y $\mu_o= 4\pi10^{-7}H/m$. Calcular la permeabilidad relativa $\mu_r$ para el problema anterior.
    \item Una chapa de silicio es sometida a un campo magnético de 9000Av/m según el cuadro 1 determinar la inducción $B$ en Teslas.
    \item Usando el cuadro 1 determinar la permeabilidad absoluta $\mu$ para Chapa normal con intensidades de campo de $H=100Av/m$, $H=1200Av/m$, $H=10000Av/m$
    \item Usando el cuadro 1 determinar la permeabilidad relativa $\mu_r$ para hierro forjado expuesto a intensidad de campo $H=120$, $H=2400$, $H=27000$ y $H=580$.
    \item Realizar un cuadro similar al cuadro 1 pero colocando las permeabilidades relativas y absolutas en 2 nuevas columnas. ¿La permeabilidad varía?
    \item Construya un  3 gráficos para hierro forjado, chapa normal y chapa silicio. Donde el eje x sea la la intensidad de campo $H[Av/m]$ y el eje Y sea la inducción magnética $B[Teslas]$
\end{enumerate}
\subsubsection{Resumen de formulas}
$ H=\frac{NI}{L}$ ; $ B=\mu H$ ; $\mathcal{F}=NI$ ;  $\mu_r=\mu/\mu_o$ ; $\mu_o= 4\pi10^{-7}H/m$ ; $ F = 40000 B^2 S$

\end{document}
