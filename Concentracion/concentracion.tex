\documentclass[12pt,openany]{book}
\usepackage[utf8]{inputenc} 
\usepackage[spanish]{babel}
\usepackage{svg}
\usepackage{amsmath}
\usepackage{geometry} 
\usepackage{enumitem}

\geometry{a4paper} 
\geometry{margin=2.2cm}
\title{Carpeta de Procesos Industriales.\\ Universidad Tecnológica Nacional.\\ Facultad Regional San Rafael.\\ \\ 2017.}
\author{Ferreyra, Gustavo David.}
    
\begin{document}
\maketitle
\tableofcontents 
\chapter {Extracción}
\section{Introducción}

\textit{Se denomina\textbf{ extracción} a la eliminación de un compuesto soluble,
presente bien en forma sólida o líquida, de un sólido o un líquido 
 por medio de un disolvente.}  Se utiliza en gran variedad de industrias,
 que van desde la extracción de oro de sus minerales, hasta la extracción de los productos 
 farmacéuticos a partir de las plantas. El disolvente no está limitado únicamente al agua, se pueden definir
 cuatro grupos.

 \begin{enumerate}
    \item \textit{Extracción de un cuerpo soluble de sólidos gruesos.} Incluye materiales que son
    suficientemente gruesos como para permitir la percolación del solvente a través de ellos. La velocidad de 
    disolución del constituyente deseado es relativamente rápida.
    \item \textit{El sólido se presenta más o menos dividido.}El líquido puede fluir a través de ellos
    fácilmente o puede ofrecer una resistencia considerable al flujo de
    líquido. Se diferencia del primer grupo en que en estos es necesaria una gran cantidad de 
    tiempo para llevar al material que se desea extraer a la superficie de las
    partículas y ponerlos en solución.
    \item \textit{Sólidos que pueden dividirse finamente}. Pueden quedar en suspensión
    permamentemente en el disolvente. Aquí en el tiempo necesario
    para la solución es indiferente, puesto que con tal que el sólido
    pueda ser puesto en suspensión el proceso continuará.
    \item \textit{Extracción de un constituyente disuelto} en el líquido que lo contiene por medio
    de otro líquido Inmiscible con el primero. 
\end{enumerate}

En algunos casos la velocidad de disolución es lo suficientemente grande, como para que en un 
solo paso del disolvente a travésdel material produzca una extracción satisfactoria. Después
de un lavado con disolvente nuevo que elimina la solución que queda adherida, se descarga el sólido. Pueden
descargarse manualmente, pero también se puede automatizar la descarga. Según éste método
la solución que se obtiene es relativamente diluída. Para obtener una solución más concentrada se puede
usar la operación en contracorriente.

Se trabaja con una batería de extracción, ingresando el agua pura en el depósito con sólido más agotado, fluye por los
diferentes depósitos conectados en serie y descarga en el que tiene una carga nueva.

El sólido en cada depósito permanece estacionario hasta que está totalmente extraído. Se puede ingresar aguapura en 
cualquiera de los tanques y extraer la solución concentrada también en cualquiera de ellos. Un ejemplo en la actualidad
es la extracción utilizada en los campos de nitrato de Chile,(extractos de taninos y lixiviado de minerales de cobre).

Para el lixiviado de sólidos granulares puede utilizarse un clasificador Dorr, se puede extraer soluto por un lavado
superficial. Se pueden usar varios equipos colocados en contracorriente.

\section{Equipos para el lixiviado de sólidos gruesos}
\subsection{Depósitos Abiertos}
El depósito abierto es la forma más primitiva y contiene un falso fondo o filtro de alguna clase. Se carga
al material sólido, y el disolvente se introduce por la parte
superior y se hace percolar y descender a través de la carga, y
se descarga por debajo del falso fondo. Aquí el disolvente llena completamente
el depósito.
El filtro inferior puede tener variedad de formas.
\begin{enumerate}
    \item Tablas perforadas sobre tirantes con muescas.
    \item Piezas triangulares sobre tirantes con muescas, rellenado el espacio con
    grava que actúa como medio filtrante.
    \item Tirantes finos soportados por otros transversales.
    \item Filtros de tela colocado sobre tirantes.
\end{enumerate}


\section{Lixiviado de sólidos intermedios}

Estos sólidos pueden ser sustancias vegetales. la sustancia a 
extraer está contenida en la extructura de la plata y para ponerla en solución se debe:
\begin{enumerate}
    \item Moler el sólido para romper las células.
    \item Dar tiempo para que el soluto pueda difundir hasta la superficie, 
    donde se podrá poner en contacto el soluto con el disolvente. Las operaciones de este tipo 
    pueden comprender a la extracción de azucar de la remolacha, y el aceite de las semillas 
    oleaginosas.
\end{enumerate}

\section{Lecho fijo o batería de difusión}

Fue desarrollado para la industria de azúcar, también en la obtención 
de taninos de las cortezas que los contienen.
Consiste en una serie de depósitos llenos de material que ha de extraerse. Las 
tuberías están distpuestas de forma tal que el agua pura se pone en contacto con 
el material más agotado y la solución concentrada se extrae en contacto con el material acabado de cargar.

cada célula de la batería se llena y descarga completamente en cada operación 
a su debido tiempo, por ello las células cambian de posición en el ciclo; las tuberías debe estar 
dispuestas para que el agua entre en cualquiera de las celdas, y el líquido concetrado pueda 
descargarse en cualquiera de ellas.
En la figura en cada célula o vaso hay un calentador, ya que el proceso se 
efectúa más rapidamente a alta temperatura. Se usan dos tuberías principales, una 
para el agua y otra para la solución. En la figura 7.2.a, la célula uno está casi 
agotada y la tres acaba de cargarse. En la célula uno fluye hacia abajo a través de 
ella y hacia arriba en el calentador, en la célula dos hacia abajo y hacia arriba en su 
calentador. En las tres se hace fluir hacia arriba el agua para eliminar el aire entre las cosetas, en 
cual escapa por una válvula (no se representa en la figura), hasta que tomando la posición indicada 
en la figura 7.2.b. Se continúa así hasta que la célula uno este completamente agotada.
Estas baterías pueden tener de 10 a 15 células. Cuando la batería se dispone 
en línea recta debe existir una tercera tubería, llamada tubería de retorno, que lleva la 
solución de un extremo a otro de la batería.

\section{Baterías de difusión contínuas}

Se ha ido reemplazando la batería anterior por otra que necesite menor 
número de obrereos y produzca más concentración en el jugo y extraiga más a las 
cosetas. Aquí se propone un movimiento contínuo de las cosetas por medios mecánicos a través de la batería. 

\section{Lixivado contínuo de sólidos finos}

El proceso de extracción comprende dos casos, que pueden efectuarse en el 
mismo equipo.
\begin{enumerate}
    \item \textit{Disolución de un compuesto} soluble de una mezcla, más o menos íntima con 
    otra constituyente. (Ejemplo: disolución de oro de sus minerales, de cobre de sus 
     minerales oxidados, disolución de azúcar contenido en células de la remolacha, disolución 
     de aceite de las células de las semillas oleaginosas, etc...). En estos casos el material 
     a obtener no sólo ha de ser disuelto, sino que la solución hay que separarla del sólido.
     \item \textit{Cuando no hay que efectuar disolución alguna}, sinó que se trata de un sólido 
     suspendido en una solución de la que debe separarse. []( A veces puede efectuarse por filtración).
\end{enumerate}

En muchos casos una reacción química produce un precipitado que debe dejarse libre de solución por 
lavado. Esto puede efectuarse agitando energéticamente el precipitado con el disolvente, dejar sedimentar
y decantar la solución que sobrenada, la operación se repite las veces que sea necesario.

Puede hacerse cada vez con una carga de disolvente nuevo, o puede hacerse en contracorriente con el 
disolvente, o sea el disolvente nuevo se puede usar únicamente para lavar el precipitado situado en la 
etapa anterior, que está proximo a ser descargado.

En las operaciones en gran escala es deseable operar en forma continua, y el sistema resultante 
recibe el nombre de decantación en contracorriente.

\section{Espesador de Dorr}

Es el aparato de sedimentación más utilizado. Consiste en un depósito de fondo plano de diámetro relativamente grande
comparado con su profundidad y que lleva un eje central sobre el que van colocados unos brazos que giran a pequeña
velocidad. la alimentación se hace en el centro con la menor perturbación posible. El diámetro del depósito se hace de la dimensión
adecuada para que el tiempo de permanencia de líquido sea suficiente para que el sólido sedimente. El líquido claro
sale por la parte superior a un canal que rodea al depósito, mientras que los brazos móviles llevan el sólido sedimentado al
centro del depósito por donde se descarga.

El espesador de Dorr es un dispositivo para separar sólidos de líquidos, pero es útil en procesos de lavado que estamos viendo
en la figura 7.11

\section {Extracción líquido-líquido.}
Aunque la extracción liquido-líquido no se emplea con tanta amplitud como la destilación, tiene importantes aplicaciones
En la industria de petróleo, para la refinación de aceite para mejorar sus características de viscocidad. También en la recuperación
de hidrocarburos aromáticos de parafinas y de hidrocarburos nafténicos.
Para eliminar los compuestos azufrados de las gasolinas. Refinación de aceites vegetales, productos farmacéuticos, etc.
Los equipos para la extracción líquido-líquido más sencillos constan de aparatos mezcladores, que consisten en tanques con dispositivos
apropiados de agitación. Deben proporcionar suficiente superficie de contacto durante un tiempo adecuado para que tenga lugar la transferencia
de soluto.
La superficie debe ser conseguida por medio de la dispersión de una fase en la otra, y su magnitud depende del número de gotas
formadas por unidad de volumen. No se debe sobrepasar un tamaño mínimo de gotas para evitar dificultades en la posterior sedimentación. También
puede separarse por centrifugación.

\subsection {Torres de extracción}

Las torres de extracción operan con las dos fases en forma líquida, moviéndose en contracorriente. la fase pesada se introduce por la cabeza
de la torre y la fase ligera por la base de la misma. La primera fluye hacia abajo y la segunda hacia arriba. Las torres de extracción pueden 
dividirse en:
\subsubsection{Torres de platos}
Los tipos más corrientes de torres de platos que se utilizan son los de platos perforados y los de platos pantallas. La construccion de los platos 
es muy similar a las columnas del mismo tipo utilizados en la destilación.
Los platos pantallas se caracterizan por el uso de pantallas horizontales para dirigir el flujo de líquido en zig-zag, tanto en el camino hacia arriba 
como hacia abajo de la torre.
\subsubsection{Torres rellenas}
Las torres rellenas se utilizan igual que para la destilación. El uso del relleno tiene a aumentar la superficie de contacto entre las fases.
\subsubsection{Torres de rociado}
Este tipos de torres de extracción es más sencilla. Son torres vacías sin relleno ni platos o pantallas. Uno de los líquidos llena completamente 
la torre como una fase continua, mientras se dispersa la otra fase a través de ella por rociado. La efectividad del contacto entre las fases es relativamente baja, 
porque tiene lugar la recirculación de la fase continua y la coalescencia de las burbujas de la fase dispersa, necesitando grandes alturas para asegurar 
la equivalencia a la etapa teórica.
Todo el equipo de extracción descrito utiliza la fuerza de la gravedad para efectuar el movimiento y la separación de las dos fases.

\subsection {Extractores centrífugos}

Los extractores centrifugos aumentan la turbulencia y el grado de contacto por el empleo de elevadas velocidades de rotación. Es particularmente 
apropiadopara fases que tienen una diferencia de densidad muy pequeña o presentan tendencia a la emulsificación. Ver el extractor de Podbielniak.
Consiste funcamentalmente en un cilindro de acero A, que contiene un cierto número de anillos concentricos de chapa de acero perforado B.
Este miembro giratorio está unido a unos muñones colocados en cojinetes de bolas C. En uno de los muñones va montada una polea motora D. El líquido 
pesado se introduce por E, pasa a través del canal F y entra en un plato rotativo por G. Como el elemento giratorio está girando de 2000 a 5000 RPM, la 
fuerza centrífuga obliga al líquido pesado a pasar por las perforaciones de los platos y a llenar el espacio H, de donde sale al exterior por los canales J y
 finalmente por las conexiones k. El líquido ligero se introduce por L, pasa por los canales M y descarga cerca de la parte exterior de la 
 sección giratoria en el espacio h. Puesto que el líquido pesado es condicido al exterior por la fuerza centrífuga, desplaza al líquido ligero,
 que fluye hacia el interior a través de los platos perforados, se recoge en el espacio N y sale al exterior a través de los canales O y conexión P.
 La posición de la superficie de contacto principal entre el líquido ligero y pesado, se controla por la presión de descarga de líquido más ligero. 
 Si los dos líquidos contienen sólidos en suspensión hay que disminuir la cantidad de los mismos por una corriente de agua que se introduce 
 por Q, y después de pasar a través de los platos, sale por los mismos canales J por donde sale el líquido más ligero. hay una serie de agujeros 
 de limpieza R, pero normalmente están cerrados por tapones roscados herméticos. 

\chapter{Extracción sólido-líquido.}
La separación de un constituyente soluble del sólido que lo contiene 
, por extracción con un disolvente, puede considerarse en dos etapas.
\begin{enumerate}
    \item \textit{Contacto} del sólido con la fase líquida.
    \item \textit{Separación} de la fase líquida del sólido.
\end{enumerate}
Estas dos etapas pueden realizarse en aparatos distintos o en el mismo.
En una operación real es completamente imposible separar la fase líquida del sólido.
Como consecuencia de ellos las corrientes resultantes que proceden de la segunda etapa serán.:
Una fase líquida (solución.) que en una operación normal no debe contener ningún sólido y un lodo constituído por el sólido y 
la solución que lleva adherida. Para designar esas dos corrientes, se utilizarán los términos \textit{flujo superior y flujo inferior}
Como en el caso del fraccionamiento se ha encontrado que es conveniente utilizar el concepto de \textit{etapa} en la ejecución de los calculos.
Una etapa consta de dos pasos indicados antes: contacto del sólido con la fase líquida y separación de flujo superior e inferior.
En el caso de la extracción solido-líquido se define la etapa ideal como una etapa en la que la solución que sale con el flujo superior, tiene la misma
composición que la solución que queda retenida por el sólido en el flujo inferior.
El uso del concepto de etapa ideal precisa utilizar la eficacia de etapa, con el objeto de obtener la relación entre las etapas ideales y las etapas reales.
Por necesidades de cálculo puede considerarse formados por tres componentes.
\begin{enumerate}
\item El soluto - componente A.
\item El sólido inerte - componente B.
\item El disolvente - componente S.
\end{enumerate}

\section{Cálculo para la extracción sólido-líquido}
Los cálculos para un sistema sólido-líquido sometido a extracción pueden basarse en la
utilización de los balances de materia y energía y en el concepto
de etapa ideal, tal como se hizo en destilación. La menor importancia de los cambios de 

energía en el proceso de extracción industrial, conduce normalmente a la omisión de las 
ecuaciones de balance de energía. Como resultado de ello los cálculo se basan en balance de materia y el concepto de etapa ideal.
Pueden utilizarse métodos de solución algebraicos y gráficos, puesto que son procedimientos equivalentes para resolver las relaciones de balances de materia y de etapa ideal.
La soluación gráfica presenta la ventaja de permitir un tratamiento generalizado de los casos más complicados y permitir una mejor vizualización de cuando va ocurriendo en el proceso,
aunque puede ser inconveniente su uso si se precisa un gran número de etapas. Puesto que la mayor parte de los casos de extracción sólido-líquido el 
número de etapas utilizadas no es muy grande la solución gráfica es el método que consideraremos.
\section{Diagrama triangular.}
El sistema de tres componentes formado por el soluto, el sólido inerte y el disolvente, a temperatura constante, puede representarse en un diagrama triangular.
(más convenientemente rectángulo).
Este es un sistema triangular de sistema ABS a temperatura constante.

\textit{Eje horizontal:} Fracción en peso de disolvente (componente S) nula. Lugar geometrico de todas las mezclas posibles del soluto(componente A) y el sólido inerte 
(componente B). Las composiciones de componente A se designan por $X_A$ y se sitúan sobre este eje.

\textit{Eje vertical:} Lugar geométrico de todas las mezclas posibles de disolvente (componente S) con el solido interte (componente B).

\textit{Hipotenusa:} Lugar de todas las mezclas posibles de disolvente(componente S) y de soluto (componente A). Las conectraciones de componente S se 
representa por $X_S$ y se sitúan sobre el eje S.
Cualquier punto interior del diagrama, punto 1 representa una mezcla de tres componentes. Si las concentraciones se representa por $X_A$,$X_B$ y $X_S$ respectivamente, 
y son fracciones en peso de cada uno se tiene.

$X_A+X_B+X_S=1$
$X_B=1-X_A-XS$

Se deduce que la composición de $X_B$ no es independiente, sinó que está determinada si se conocen $X_A$ y $X_B$.

Cualquier punto puede situarse en el diagrama trazando las coordenadas $X_A$ y $X_S$ únicamente. El origen $X_A=0$ y $X_S=0$ representan el sólido inerte. Las 
líneas paralelas a la hipotenusa del triángulo son líneas de composición constante en sólido interte ($X_B=C$); puesto que estas líneas tienen una ecuación de la forma:

$X_S=-XA+D$ ; $X_S+X_A=D$; $D=constante$
De arriba : $X_B=1-X_A-XA=1-D=C$
La solución satura del componente A en el disolvente, a la temperatura considerada, está representada por el punto 2 del diagrama.
La región por encima de la línea que une el origen con el p unto 2, representa la sección del diagrama en el que no hay componente de A no disuelto y es la región en la que se efectúan la mayor parte de las extracciones.
La región por debajo de la línea antes citada representa la sección del diagrama en la que los sólidos presentes contienen los dos componentes B y el A no disuelto y la solución presente es solución saturada con la composición representada por el punto 2.
Si el soluto estuviera presente como líquido y no es completamente miscible con el disolvente, en este caso pueden presentarse dos soluciones saturadas; una formada por el componente S saturado con A y otra el componente A saturado con S.
\section{Solución Gráfica}
Consideremos un sistemad e una sola etapa en condiciones de régimen permantente (figura).

$V_2$:Velocidad de alimentación de disolvente Kg/h.
$V_1$: Caudal de la corriente de flujo superior Kg/h.
$L_0$: Velocidad de alimentación de la etapa.
$L_1$: Caudal del flujo inferior que sale de la etapa.



\section {Cálculo de extracción en contracorriente con etapas múltiples.}


Este es un sistema esquemático de un sistema de extracción en contracorriente con etapas múltiples que contiene $n$ etapas ideales. La velocidad de alimentación se representa por $L_0$ y la del disolvente suministrado por $V_{n+1}$. Por la etapa 1 representa el extremo del sistema por el que se introduce la alimentación y desde el cual sale el flujo superior de concentración más elevada del soluto. la etapa $n$ representa el extremo del sistema por el que se efectúa la alimentación del disolvente puro y del que sale el flujo inferior con la concentración más baja en soluto.
  
Vemos que el sistema es análogo al de una columna de destilación en platos debajo de la alimentación (sección de agotamiento).

Consideraremos que conocemos $L_0$ y su composición $X_0$, $V_{n+1}$ y su composición $Y_{n+1}$ y el contenido de soluto del flujo inferior que sale del sistema  $X_{(A)n}$. Consideraremos también que disponemos de información para determinar el lugar geométrico de las composiciones de flujo inferior. Veamos la solución gráfica para obtener el números de etapas teóricas.
\begin{figure}
    \center
    \tiny  
    \includesvg[scale=0.6,extractwidth=300pt]{dibujo}
    \caption{Esquema de etapa múltiple de concentradores}
\end{figure}
    
\section{Balance de materia y componentes.}
\newcounter{neq}
Balance de materia sobre el total del sistema es:

\begin {equation}
L_0 + V_{n+1} = L_n + V_1
\addtocounter{neq}{1}
\end{equation}

\begin {equation}
L_0 X_0+ V_{n+1}Y_{n+1} = L_n X_n + V_1  V_1
\addtocounter{neq}{1}
\end{equation}

\begin{figure}
\center
\includesvg[scale=1.8]{grafico}
\caption{Solución gráfica para la extracción con múltiples etapas.}
\end{figure}

Si llamamos $M$ a la suma de las corrientes $L_0$ y $V_{n+1}$, el punto $X_M$ que representa la composición de $M$, puede situarse sobre el diagrama, a partir del hecho de que debe estar situado sobre la línea recta que une los puntos $X_0$ e $Y_{n+1}$, y que su posición sobre la línea se determina por la relación $V_{n+1}/L_0$.

De las ecuaciones $(1)$ y $(2)$, $M$ también es la suma de las corrientes $L_n$ y $V_1$. Por consiguientes los puntos $M_n$ y $M_m$ $Y_1$ deben estar situados sobre la misma recta. Puesto que se conoce el lugar geométrico de las composiciones de flujo inferior, el punto $X_n$ se sitúa en el l ugar geométrico con el valor $X_{(A)n}$.

El punto $Y_1$ se contruye la recta que une $X_n$ y $X_M$ y determinando su intersección con el lugar geométrico de las composiciones de flujo superior. En este momento se conoce las composiciones de todas las corrientes terminales. Si escribimos las ecuaciones de balance de materia para la parte del sistema desde la etapa 1 hasta la $j+1$, y se simplifica:

\begin {equation}
L_0 - V_{1} = L_j - V_{j+1}
\addtocounter{neq}{1}
\end{equation}

\begin {equation}
L_0 X_0 - V_{1}Y_{1} = L_j X_j - V_{j+1}Y_{j+1}
\addtocounter{neq}{1}
\end{equation}

En las que la etapa $j$ puede ser una etapa cualquiera del sistema ($j=1,2,3...,n$). Hagamos que R quede definida por:

$R= L_0 - V_1$
$R X_R = L_0 X_0-V_1 Y_1$

Entonces las ecuaciones $(3)$ y $(4)$:

\begin {equation}
R = L_0 - V_1=L_1-V_2= ... =L_n-V_{n+1}
\addtocounter{neq}{1}
\end{equation}

\begin {equation}
R X_R = L_0 X_0 - V_1 X_1=L_1 X_1-V_2 Y_2= ... =L_n X_n-V_{n+1} Y_{n+1}
\addtocounter{neq}{1}
\end{equation}
 
El punto $X_R$ se localiza por la intersección de las rectas que pasan por los puntos $X_0$,$Y_1$ y $X_n$, $Y_{n+1}$ , puesto que según las ecuaciones 5 y 6 ha de pertenecer a ambas.
La solución gráfica puede comenzarse en cualquiera de los extremos del sistema. Si empezamos por la etapa 1, como el punto $Y_1$ es conocido, el $X_1$ se determina usando la relación de etapa ideal, es decir construyendo la línea que pasa por el origen (si todo el soluto está disuelto) y el punto $Y_1$ determinando su intersección con la línea de composición de flujo inferior. Como $X_1$ y $X_R$ son conocidos puede determinarse $Y_2$ por la intersección de la línea que pasa por $X_1$ y $X_R$ y la línea de composiciones de flujo superior, ya que
 
$R=L_1 -V_2$

$R X_R = L_1 X_1 - V_2 Y_2$.

El procedimiento puede repetirse hasta que el valor obtenido de $X_A$ sea menor o igual que $X_{(A)n}$. El número de etapas ideales necesarias es igual al número de rectas de reparto, líneas que pasan por el origen y los puntos $Y_1$, $Y_2$, etc. Para este gráfico es 2.

\textit{Ejemplo:} Se va a extraer aceite de una harina por medio de benceno, usando un extractor continuo en contracorriente. La unidad va a tratar 2000Kg de harina por hora (basado en sólido completamente agotado). La harina no tratada contiene 800Kg de aceite y 50kg de benceno. la solución extractora nueva está formada por 1310kg de benceno y 20kg de aceite. los sólidos agotados contienen 120kg de aceite no extraído. Experimentos realizados en condiciones identicas a aquellas para las que se proyecta la batería, indican que la solución retenida por el sólido depende de la concentración de la solución.

Encontrar:

\begin{enumerate}[label=\alph*)]
\item La composición de la solución concentrada. 
\item La composición de la solución que queda adherida al sólido extraído.
\item El peso de la solución que sale de la harina extraída.
\item El peso de la solución concentrada.
\item El número de unidades necesarias.
\end{enumerate}

Solución.

\textit{Base del cálculo 1 hora}

$L_0=2000+800+50 = 2850kg$

$X_{A0}=800/2850=0,281$

$X_{S0}=50/2850=0,01754$

$V_{n+1}=20+1310=1330 kg$

$Y_{A n+1}=20/1330=0,0150$

$Y_{B n+1}=20/1330=0$

$Y_{S n+1}0,985$

Sea $M = L_0+V_{n+1}=2850+1330 =4180$

$\frac{L_0}{M} = \frac{2850}{4180}$

\end{document}
