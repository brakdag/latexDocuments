\documentclass[11pt,openany]{book}
\usepackage[margin=14mm, a5paper]{geometry}
\usepackage[utf8]{inputenc}
\usepackage[spanish]{babel}
\usepackage{trees}
\usepackage{svg}  
\usepackage{amsmath}
\usepackage{geometry} 


\title{DESTILACIÓN}
\author{Gustavo David Ferreyra}
\newtheorem{defi}{Definición}

\begin{document}
\makeatletter
\begin{titlepage}
\centering
\includesvg[scale=1,extractwidth=50pt]{./img/utn} \\
\vspace{1cm}
\begin{Huge}
\@title \\ 
\end{Huge}
\vspace{1cm}
\begin{large}
Universidad Tecnológica Nacional \\ 
  Facultad Regional San Rafael. \\
\end{large}
\vspace{5cm}
\@author \\
\@date \\
\end{titlepage}
\newpage

\newpage\null\thispagestyle{empty}
\newpage\tableofcontents{\thispagestyle{empty}}
\newpage\setcounter{page}{1}
\chapter{DESTILACIÓN}
\section{Introducción}
\begin{defi}[Destilación]
La destilación es un proceso que consiste en calentar una mezcla líquida, hasta que sus componentes más 
volátiles pasen a la fase de vapor y a continuación, enfriar el vapor para recuperar dichos 
componentes en forma líquida por medio de la condensación.
\end{defi}

El objetivo principal de la destilación es separar una mezcla de varios componentes aprobechando 
sus distintas volatilidades, o bien separar los materiales volátiles de los no volátiles.

En los procesos de evaporación y secado, normalmente el componente más volátil es agua que se desecha, porque el objetivo es obtener 
el componente menos volátil. Por el contrario, en la destilación la finalidad principal es obtener el componente más volatil.

Por ejemplo, la eliminación de agua de la glicerina evaporando el agua, se llama evaporación, pero 
la eliminación de agua del alcohol, evaporando el alcohol se llama destilación, aunque se usan 
mecanismos similares en ambos casos.

El proceso de destilación, consta de dos fases:
\begin{enumerate}
\item El líquido pasa a vapor. 
\item El vapor se condensa, volviendo a estado líquido en un matráz distinto de donde se produjo la evaporación.
\end{enumerate}
\begin{figure}
  \centering
  \includesvg[scale=0.2,extractwidth=200pt]{./img/destila}
  \caption{Destilación}  
\end{figure}
 
La diferencia entre destilación y evaporación es la siguiente:


\textbf{Destilación:} La destilación es el procedimiento más utilizado para la separación y purificación de 
líquidos, se utiliza siempre  que se protende separar un líquido de sus impurezas no volátiles.
La exigencia básica para la separación de los componentes de una mezcla por destilación es que la 
composición del vapor sea diferente a la composición del líquido con el cual se encuentra en equilibrio. Si 
la composición de ambos es igual, nunca podrá separarse por destilación.

\textbf{Evaporación:} Consiste en calentar la mezcla hasta el punto de ebullición de uno de los 
componentes, y dejarlo hervir hasta que se evapore totalmente. Se emplea si no tenemos interés 
en utilizar el componente evaporado. Los otros componentes quedan en el envase.

\subsection*{Ejemplos}

\textbf{Desalinización del agua}, la desalinización es un proceso que utiliza el principio de destilación para 
extraer la sal del agua de mar. El agua se calienta y se bombea a un tanque de baja presión, donde 
se evapora parcialmente. A continuación, se condensa el vapor formado por esta ``evaporación súbita'' y se 
extrae como agua pura.

Si la diferencia entre las temperaturas de ebullición o volatilidad de las sustancias es grande, se 
puede realizar fácilmente la separación completa en una sola destilación. Es el caso de la obtención de
agua destilada a partir del agua marina. Esta contiene aproximadamente el 4\% de distintas materias 
sólidas en disolución.

\textbf{La separación de agua y alcohol}, es un ejemplo de proceso de destilación. Cuando se lleva a ebullición 
una disolución de alcohol, la mayor parte del vapor inicial es de alcohol, pues alcanza su punto de 
ebullición antes que el agua, el vapor se recoge y se condensa varias veces para obtener la mezcla de alcohol
más concentrada. Veamos en detalle, donde el agua hierve a $100^oC$, y el alcohol, que hierve a $78,5^oC$. Si 
se hierve una mezcla de estos dos líquidos, el vapor que sale es más rico en alcohol y más pobre en 
agua que el líquido del que precede, pero no es alcohol puro. Con el fin de concentrar una 
disolución que contenga un 10\% del alcohol (como la que puede obtenerse por fermentación) para 
conseguir una disolución que contenga un 50 \% de alcohol. (frecuentemente en whisky), el destilado ha 
de volver a destilarse una o dos veces mas, y si se desea alcohol industrial (95\%) son necesarias 
varias destilaciones.

\textbf{Solución de glicerina diluída}: si se desea obtener glicerina, se separa por evaporación hasta la 
concentración de 80\% de glicerina (se genera vapor de glicerina) donde se inicia la destilación.

\section{Diagrama de puntos de ebullición}
\begin{figure}[h]
\includesvg[scale=0.4,extractwidth=20pt]{./img/image}
\caption{Relación entre el punto de ebullición y la composición de equilibrio a presión constante para todas las 
mezclas de un líquido $A$ (más volátil) y otro $B$, con puntos de ebullición $t_A$ y $t_B$.
}
\end{figure}

La concentración $X$ de un componente $A$ de una mezcla se calienta, comienza a hervir a $t_1$ y, la 
concentración del primer vapor será $Y$, por lo que el líquido tiene una composición menor del componente más volátil.
La mezcla representada en el diagrama de punto de ebullición sigue la ley de Raoult.

\section{Ley de Raoult}

En algunos casos es posible obtener el diagrama de los puntos de ebullición a partir de los datos de
 la presión de vapor de los componentes puros. Esta ley establece que: 
\begin{defi}[\textit{Ley de Raoult}]
 A temperatura constante la presión parcial de un 
 componente de la mezcla es igual 
 a su fracción molar multiplicada 
 por la presión del vapor del mismo componente 
 en estado puro a la temperatura considerada.
\end{defi}
 La presión del vapor o más comúnmente presión de saturación es la presión de la fase gaseosa o 
 vapor de un sólido o un líquido sobre la fase líquida, para una temperatura determinada, en la que 
 la fase líquida y el vapor se encuentran en equilibrio dinámico, su valor es independiente de las 
 cantidades de líquido y vapor presentes mientras existan ambas.

 Según esto la presión parcial de un componente varía linealmente desde cero hasta el valor de su 
 presión de vapor puro a medida que su fracción molar varía de cero a la unidad.

 \begin{figure}[h]
  \centering
  \includesvg[scale=0.4,extractwidth=20pt]{./img/image2}
  \caption{Diagrama benceno-tolueno}  
\end{figure}
  
 Consideramos el sistema benceno-tolueno. A la temperatura de $t_A=100^oC$ el tolueno tiene una presión 
 de vapor de $P_A=556 mmHg$. La gráfica en función de la composición será una recta. A ésta misma 
 temperatura el benceno tiene una presión de $P_B=1350 mmHg$, y su presión de vapor varía de 0 para el
  tolueno puro hasta $P_B=1350 mmHg$ de benceno puro. La presión vapor total para cualquier composición será 
  la suma de las dos rectas.\footnote {Si son rectas es aplicable la ley de Raoult.}

  En el punto en el que la presión total es $P=760 mmHg$ se encuentra que las fracciones molares en 
  el líquido son $X=0,263$ para el benceno $1-X=0,737$ para el tolueno. La composición para el vapor se 
  determina por medio de la ley de Dalton. Aquí $P_A=351 mmHg$ y la presión total $P=760 mmHg$, por lo tanto la 
  composición del vapor y fracción molar del benceno es $Y_A=P_A/P=351mmHg/760mmHg=0,46$.
  
 Se puede expresar la ley de Raoult:
\begin{equation}\rho_A=P_A X\end{equation}
donde:
\begin{description}
  \item [$\rho_A=$] {Presión parcial del componente $A$ sobre la solución con fracción molar $X$.} 
  \item [$P_A=$] {Presión de vapor componente puro $A$.}
  \item [$P_B=$] {Presión de vapor componente puro $B$.}
  
\end{description}
Para el segundo componente de la mezcla:
\begin{equation}\rho_B=P_B (1-X)\end{equation}

En la línea trazada en la figura anterior podemos saber las presiones parciales de los diferentes
 componentes por medio de la ley de Raoult:

$\rho_A=0,263\ 1350mmHg = 355mmHg$

$\rho_B= 0,737\ 566mmHg = 409mmHg$

$P = 355mmHg + 409mmHg = 764 mmHg$ 
Si $P$ es la presión total:
\begin{equation}
P=\rho_A + \rho_B = P_A X+P_B (1-X)
\end{equation}
como $Y$: fracción molar de $A$ en el vapor es igual al cociente de la presión parcial de A a la presión total:
\begin{equation}
Y=\frac{\rho_A}{\rho_A+\rho_B}=\frac{P_A X}{P_A X+P_B (1-X)}=\frac{P_A X}{P}
\end{equation}
La ley de Raoult se aplica únicamente a las mezclas de sustancias que son químicamente muy similares 
y cuando las moléculas de las dos sustancias no tienen acción alguna entre si. Ejemplo de la 
divergencia de la ley de Raoult, consideramos la mezcla de sulfuro de carbono-acetona.

\begin{figure}[h] \label{fig:car-acet}
  \centering
  \includesvg[scale=0.4,extractwidth=20pt]{./img/carb-ace}
  \caption{Diagrama sulfuro de carbono - acetona}  
\end{figure}
 

\section{Ley de Henry}

Análoga a la ley de Raoult. La presión parcial de un componente en la solución es proporcional a su 
fracción molar en el líquido.
\begin{equation}\rho_A=c X\end{equation}
Donde:
\begin{description}
\item [$\rho_A=$] Presión parcial de un componente $A$ sobre una mezcla líquida.
\item [$c=$] Constante a temperatura constante.
\item [$X=$] Fracción molar de $A$.
\end{description}
La figura anterior muestra una curva de presión parcial para cada uno de los componentes de la 
mezcla, la cual es sustancialmente una línea recta en el final de la curva, en que el componente en 
cuestión está en pequeñas cantidades.

\section{Mezclas de punto de ebullición constantes}

Diagrama correspondiente a cloroformo-acetona. La composición de \textbf{a} tiene el punto de ebullición más elevado.

El diagrama tiene el mismo punto de ebullición y corresponde a la temperatura \textbf{b} y la composición \textbf{a}.
En ambos diagramas \textit{las curvas de líquido y vapor son tangentes} al punto de ebullición mínimo o mácimo.
La composición de vapor que se desprende de una mezcla de punto de ebullición máximo o mínimo es la misma que la del líquido
hirviente; es decir el punto de ebullición es constante.
Estas mezclas no se pueden separar por destilación.

Si la presión sobre las mismas varía, se puede cambiar el punto de ebullición constante y esto puede utilizarse para 
efectuar separaciones, bien a presión o vacío.

\section{Equilibrio de composición}

La forma simplificada de los diagramas anteriores (Cloroformo y Benceno) y reciben el nombre de diagramas de equilibrio 
líquido vapor (reacción entre composición del líquido y el vapor a presión constante)

El segundo diagrama de equilibrio se respresenta la mezcla cloroformo-acetona,
 en donde la curva de equilibrio corta a la diagonal trazada por el origen, 
 la intersección corresponde al punto de ebullición constante.

La construcción de la curva líquido-vapor es fácil cuando se dispone del diagrama punto de
 ebullición. Primer se elige una composición, se traza una vertical hasta 
 cortar con la línea inferior o de líquido; desde ese punto trazar la horizontal hasta cortar la línea 
 superior o de vapor y desde aquí una vertical hacia abajo, obteniéndose la composición del vapor 
 que está en equilibrio con el líquido de partida.

 \section{Métodos de destilación}

 Hay dos métodos principales:
 \begin{enumerate}
 \item Cuando se produce un vapor por ebullición de la mezcla líquida que se quiere separar, condensando esos vapores.
 \item Cuando se envía parte envía parte del condensado al calderín (columna de destilación) y este retorno 
 se pone en intimo contacto con los vapores que se desprenden en contracorriente y van hacia el condensador (rectificación).
 \end{enumerate}

 \begin{figure}
  \tree Destilación
  \subtree Simple \endsubtree
  \subtree Relámpago \endsubtree
  \subtree Fraccionada
       (Columna de
        Destilación) 
          \subtree Contínuo
           (relleno) 
          \subtree Anillo 
          rasching \endsubtree
          \endsubtree
          \subtree Por etapa
           (plato) 
          \subtree Plato 
          perforado \endsubtree
          \subtree Campana
           burbujeadora \endsubtree
          \endsubtree
      
  \endsubtree
  \subtree
  Al Vacío
  \endsubtree
  \endtree
  \caption{Métodos de destilación}
  \end{figure}
  
\subsection {Destilación de equilibrio o relámpago}

En este método se realiza la vaporización de una fracción determinada de una carga de líquido,
manteniendo tando el líquido como el vapor formado en íntimo contacto hasta el final de la 
operación, retirando el vapor y condensándolo.

Si tenemos un sistema formado por los componentes $A$ y $B$, siendo $A$ el más volátil se puede decir que:

El número de moles contenidos en la carga $W_o$ y la composición de $A$ es $X_o$, si suponemos que se 
vaporizan V moles, quedando en el líquido ($W_o - V$) moles. La composición del líquido es $x$ y la del 
vapor en equilibrio con el es $y$.
\begin{equation}W_o X_o = V y +(W_o -V) x\end{equation}
Este método es utilizado con sistemas multi-componentes.

\subsection {Destilación simple}

Es una operación en la cual se produce la vaporización de un material por la aplicación de calor. Los 
vapores que se desprenden se eliminan contínuamente, se condensan y se recolectan sin permitir 
que tenga un lugar ninguna condensación parcial ni retorno al recipiente en donde se lleva a cabo 
el calentamiento y la ebullición. el destilado puede recolectarse en varios lotes separados
 (fracciones) en función de su grado de pureza.

 Este método se aplica en industrias de capacidad moderada y pequeña.

 \subsection{Destilación fraccionada o rectificación}

En este proceso, la mezcla líquida a separar (alimentación) se lleva al fondo de la columna (calderín) y 
se pone allí en ebullición. El vapor generado sube por la columna, sale de ella por la parte superior y 
se condensa (serpentin). Una parte del condensado se separa como producto de cabeza (destilado) y la otra 
vuelve como reflujo a la columna, por la que desciende como fase líquida.

En el interior de la columna se pone en contacto el vapor ascendente con el líquido descendente. 
En un nivel dado de la columna estas dos corrientes no se encuentran en equilibrio por lo que hay 
una transferencia de materia pasando los componentes más volátiles del líquido al vapor, y los 
componentes menos volátiles del vapor al líquido. Por lo que el vapor se enriquece de los 
elementos más volátiles a medida que asciende por la columna.

\section {Columnas de destilación}

Los distintos tipos de destilación se suelen llevar a cabo en columnas de destilación.

Estos recipientes cilíndricos verticales con una entrada de la corriente de alimentación a 
destilar por un punto dado de la columna, y una salida por la parte superior o cabeza para extraer 
los vapores a condensar. Estos pueden volver en parte a la columna como reflujo a través de otra entrada por la 
parte superior.

Para asegurar un adecuado contacto entre el vapor y el líquido (escencial para la transferencia de materia) se han 
diseñado varios dispositivos basados en dos criterios distintos:
\begin{enumerate}
\item Columnas de contacto contínuo entre el vapor y el líquido o columnas de relleno.
\item Columnas de contacto por etapas, o columnas de platos o pisos.
\end{enumerate}
\subsection{Columnas de relleno}
Se diferencian de las anteriores por tener su interior ocupado por un relleno de anillos
llamados anillos de rashing o de material perforado.
En esta columnas lo más importante es la mayor superficie de contacto ofrecida por el relleno.
En las columnas se produce habitualmente un movimiento a contracorriente entre el líquido que 
desciende y el vapor que asciende. Y es durante este movimiento cuando se purifican los 
componentes hasta la calidad deseada en el diseño.
\subsubsection{Anillos de rashing}
Son piezas de geometría tubular cuyo diámetro es aproximadamente igual a su longitud y que se 
emplean como relleno para columnas en procesos de destilación.
Generalmente se frabrican con material cerámico o metálico y poseen una elevada superficie de 
contacto, lo que facilita la interacción entre una fase líquida estacionaria y una fase móvil gaseosa.
\subsubsection{Funcionamiento}
En la columna de destilación, el reflujo o vapor condensado gotea sobre el interior de la columna.
recubriendo los anillos de Rasching.
Por otro lado, el vapor producido por la calefacción del dispositivo asciende por la columna.
De esta forma, el vapor generado y el líquido condensado interaccionan en contracorriente en el 
interior de los anillos de Rashing, tendiendo a establecer un equilibrio. Los componentes menos 
volátiles se desplazan con el líquido condensado y los más volátiles se desplazan con el vapor generado, 
es decir siguen sentidos opuestos.
\subsection{Columnas de plato}
Los platos son superficies planas que dividen la columna en una serie de etapas.
Tienen por objeto retener una cierta cantidad de líquido en la superficie a través del cual se hace 
burbujear el vapor que asciende de la caldera, consiguiendo así un buen contacto entre el vapor y el líquido.
El líquido de un platoi cae al plato siguiente por un rebosadero situado en un extremo del plato.
Según la forma del dispositivoi que permite el paso del vapor a través del líquido, se distiguen entre:
platos perforados con simples agujeros, platos de campana, de válvulas y otros.
El vapor que llega a un plato por debajo, y el líquido que llega por encima, no están en equilibrio. 
En el plato tiene lugar la mezcla íntima de ambas corrientes, produciéndose allí la transferencia de materia.
El vapor que abandona el plato hacia arriba es más rico en el componente más volátil, y en el líquido que 
cae del plato es más rico en el componente menos volátil. 

-completar-

\section{Cálculo de columna de fraccionamiento}
Cálculo para el dimensionamiento de columna de fraccionamiento por destilación.
Ingresa una mezcla $F(A Y B)$ componente binario. Que contiene una fracción de peso $X$,
del componente $A$.
La columna trabaja a una presión $P$.
El producto de cabeza o destilado ha de contener $X_o$ y el de cola o residuo $X_w$
(fracciones en peso) del componente A.

\subsection{Balance de materia y calor}
Los caudales de las corrientes terminales $D$ y $W$ se pueden calcular por los balances de materia 
(en estado de régimen).

\subsubsection{Balance de materia}
 
\begin{equation} \label{eq:total}
F=D+W   
\end{equation}
\begin{equation} \label{eq:componentes} 
F X_F = D X_D + W X_W 
\end{equation}

La ecuación~\ref{eq:total} es de balance total y la equación~\ref{eq:componentes}
balance por componentes. Reemplazando~\ref{eq:total} en~\ref{eq:componentes}: 
\begin{equation*}F X_F = D X_D + (F-D) X_W\end{equation*}
\begin{equation} D=F \frac{X_F - X_W}{X_D-X_W}\label{eq:destilado}\end{equation}
\begin{equation}W=F \frac{X_D - X_F}{X_D-X_W}\label{eq:residuo}\end{equation}

  La ecuación~\ref{eq:destilado} corresponde a la cantidad de destilado y 
  la ecuación ~\ref{eq:residuo} corresponde a la cola o residuo del destilado.
  
\subsubsection{Balance de energía}
\begin{equation}F h_F +q_r = D h_D + W h_W + q_c\end{equation}


\begin{description}
\item [$q_r=$] {Calor entregado en el hervidor $(kcal/h)$.}
\item [$q_c=$] {Calor eliminado en el condensador $(kcal/h)$.}
\end{description}
\subsubsection{Balance de materia y energía para el condensador}
$V_1 = L_0 + D$ (Balance de materia.)
\newline\newline
$V_1 Y_1= L_0 x_0 + D X_D$ (Balance de un componente)
\newline\newline
$V_1 h_1 = qc + L_0 h_0 + D h_0$ (Balance de energía)
\newline\newline
Si se condensa totalmente todo el vapor que entra, entonces
\newline\newline
$Y_1 = X_0 = X_D$
Despejando $qc$ del balance de energía.
$qc = V_1 h_1 - (L_0+D)h_D$
\newline\newline
sustituyendo $V_1=L_0 +D$
\newline\newline
$qc=(L_0+D)h_1 - (L_0+D)h_D = (L_0+D)(h_1-h_D)$
\newline\newline
Dividiendo por $D$
\newline\newline
$qc/D = (L_0/D + D/D)(h_1-hD)=(L_0/D +1)(h_1-h_D)$
\newline\newline
La relación $L_0/D$ se denomina \textit{relación de flujo externo} y al fijar este valor es equivalente a 
fijar $qc$. Puede calcularse el calor entregado en el hervidor por medio de 
$F h_F + qr = D h_D + W h_W + qc$ . En este punto tenemos fijadas todas las condiciones terminales y los valores 
desconocidos se han calculado por medio de los balances de materia y energía.
\subsubsection{Cálculo plato a plato}
Para efectuar una separación es necesario saber el número de platos en la columna.  
Uno de los métodos consite en el cálculo plato a plato partiendo de datos terminales conocidos.
Cualquiera de los dos extremos de la columna puede elegirse para comenzar los cálculos. Si se 
empieza por la cabeza de la columna, se mantiene para el primer plato.
\newline
Balance de materia (total): $L_0+V_2=L_1+V_1$
\newline \newline
Balance del componente A: $L_0 X_0 + V_2 Y_2 = L_1 X_1 + V_1 Y_1$
\newline \newline
Balance de energía: $L_0 h_0+V_2 Y_2=L_1 h_1+V_1 H_1$
\newline\newline
Cantidades conocidas $L_0$,$X_0$,$h_0$ (reflujo) y $V_1$,$Y_1$,$ H_1$ (vapor)
\newline\newline
En consecuencia existen 6 incógnitas y 3 ecuaciones, haciendo necesario conocer más datos 
adicionales, se introducen en el concepto de plato teórico.

\subsection{Plato teórico}
\begin{defi}
Se define como un plato que está cargado de cierta cantidad de líquido, del cual se desprende 
un vapor cuya composición media está en equilibrio con un líquido cuya composición es la media de 
la del líquido que sale del plato.
\end{defi}
Utilizando estos conceptos se puede hacer el cálculo del número de platos teóricos.

La composición del vapor que sale del plato se conoce, por lo tanto la composición de la corriente 
$L_1$ puede determinarse por la relación de equilibrio a la presion $P$. La composición $X_1$ de la corriente 
$L_1$ es conocida ahora, lo cual fija su entalpía. Además la corriente $V_2$ es vapor saturado por lo que 
su entalpía queda definida por su composición. De esta forma se conocen $X_1, h_1, H_2$ y el número 
de variables desconocidas queda reducido a tres ($L_1,V_2,Y_2$) el sistema puede resolverse. Al llegar a
este punto las composiciones y entalpías de las corrientes $V_2$ y $L_1$ han quedado determinadas y se 
dispone de datos suficientes para empezar los cálculos del segundo plato teórico. Es posible 
proceder de esta manera hasta el plato de la alimentación, pero el cálculo es pesado.

Una solución más rápida de los balances de materia y energía puede obtenerse gráficamente 
utilizando los diagramas entalpía-composición.
\end{document}