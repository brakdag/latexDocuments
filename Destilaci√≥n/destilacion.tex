\documentclass[a4paper,12pt]{book}
\usepackage[utf8]{inputenc}
\usepackage[spanish]{babel}
\title{Destilación \\ UTN FRSR \\ Argentina.}
\author{Gustavo David Ferreyra}
\begin{document}
\maketitle
\chapter{Destilación}
Destilación, proceso que consite en calentar una mezcla líquida hasta que sus componentes más 
volátiles pasan a la fase de vapor y, a continución, enfriar el vapor para recuperar dichos 
componentes en forma líquida por medio de condensación.

El objetivo principal de la destilación es separar una mezcla de varios componentes aprobechando 
sus distintas volatilidades, o bien separar los materiales volátiles de los no volátiles.

En la evaporación y en el secado, normalmente el objetivo es obtener el componente menos volátil;
el componente más volátil, casi siempre agua, se desecha. Sin embargo, la finalidad principal 
de la destilación es obtener el componente más volatil en forma pura.

Por ejemplo, la eliminación del agua de la glicerina evaporando el agua, se llama evaporación, pero 
la eliminación del agua del alcohol evaporando el alcohol se llama destilación, aunque se usan 
mecanismos similares en ambos casos.

La destilación, como proceso, consta de dods fases: en la primera, el líquido pasa a vapor y en la 
segunda el vapor se condensa, pasando de nuevo a líquido en un matraz distinto al de la destilación.

La diferencia entre destilación y evaporación es la siguiente:
\textit{Destilación:} La destilación es el procedimiento más utilizado para la separación y purificación de 
líquidos, se utiliza siempre  que se protende separar un líquido de sus impurezas no volátiles.
La exisgencia básica para la separación de los componentes de una mezcla por destilación es que la 
composición del vapor sea diferente a la composición del líquido con el cual se encuentra en equilibrio. Si 
la composición de ambos es igual, nunca podrá separarse por destilación.
\textit{Evaporación:} Consiste en calentar la mezcla hasta el punto de ebullición de uno de los 
componentes, y dejarlo hervir hasta que se evapore totalmente. Se emplea si no tenemos interés 
en utilizar el componente evaporado. Los otros componentes quedan en el envase.

Ejemplos:

La separación de alcohol de agua es un ejemplo de proceso de destilación. Cuando se lleva a ebullición 
una disolución de alcohol, la mayor parte del vapor inicial es de alcohol, pues alcanza su punto de 
ebullición antes que el agua. el vapor se recoge y se condensa varias veces para obtener la mezcla de alcohol
más concentrada.
\end{document}
