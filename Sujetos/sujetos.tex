\documentclass[a4paper,12pt,spanish]{book}
\usepackage[utf8]{inputenc}
\usepackage[spanish]{babel}
\usepackage{cite}
\begin{document}
\chapter {Conceptos Generales.}


Se suele llamar \textit{sujeto de la educación} al individuo que va a ser educado o al que se le va a enseñar:
un sujeto supuestamente preexistente a la relación educativa, estudiado por la psicología evolutiva y 
en general por las ciencias de la educación. Esta ide usual tiene que ser discutida si se pretende abrir un 
campo de problematización en torno al sujeto. En este documento se asume la persepectiva de que cada sujeto 
es una multiplicidad infinita cuya subjetivación depende de ciertas cirscunstancias: se es sujeto en situación y 
de la situación. ``El sujeto de la educación es un sujeto fundamentalmente colectivo porque surge de una combinación 
de distintos elementos, sin los cuales no sería posible ( maestros, estudiantes, conocimientos, prácticas.) Por lo 
tanto, no hay un sujeto preexistente, sino que hay un sujeto de y en la situaciones educativas."  La constitución 
subjetiva no es previsible, está librada al azar del encuentro, al que no preexiste. Ello implica tomar en serio el 
caracter productivo del sistema escolar, poniendo en el centro de los procesos de constitución subjetiva la historia 
del dispositivo escolar y la naturaleza del proyecto escolar, y entendiendo la escolarización como parte del diseño del 
desarrollo humano historicamente producido, por lo tanto contingente.

\section {El sujeto como problema.}

La cuestión del sujeto está atravesada por los debates teóricos y políticos que sucita la crisis de la noción
moderna de sujeto. En el mundo metateórico de la modernidad, el sujeto consituía el núcleo duro de una identidad que 
se reconocía a si misma como tal, que se diferenciaba del objeto que tenía enfrente y prescindía de cualquier otredad. Esta 
noción de sujeto moderno está firmemente apoyada sobre el \textit{cogito cartesiano:} un sujeto ``unitario, autocentrado,
racionall, consciente de sí mismo" (pineau, 1990:40)
El discurso de la modernidad ha ofrecido distintos modos la imagen de un ser consciente, racional y libre,
cuya razón lo llevaría a controlar el mundo natural, disminuyendo la incertidumbre (beck,2006)
La mirada de la pedagogía sobre los alumnos es tributaria de esa noción moderna de sujeto. Señala Pianeau que 
la modernidad convierte a la educación en un proceso por el cual el hombre se vuelve hombre, sujeto moderno,
cartesiano, con las características antes apuntadas (Pineau,1999). La educación tenía ademas la expectativa de eliminar
las limitaciones derivadas del nacimiento y de permitir que los humanos.
\cite{cerletti2008}
\bibliographystyle{plain}
\bibliography{biblio}



\end {document}