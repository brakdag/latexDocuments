\documentclass[a4paper,12pt,spanish]{book}

\usepackage[utf8]{inputenc}
\usepackage[spanish]{babel}
\usepackage{natbib}

\begin{document}
\chapter {Conceptos Generales.}


Se suele llamar \textit{sujeto de la educación} al individuo que va a ser educado o al que se le va a enseñar:
un sujeto supuestamente preexistente a la relación educativa, estudiado por la psicología evolutiva y 
en general por las ciencias de la educación. Esta ide usual tiene que ser discutida si se pretende abrir un 
campo de problematización en torno al sujeto. En este documento se asume la persepectiva de que cada sujeto 
es una multiplicidad infinita cuya subjetivación depende de ciertas cirscunstancias: se es sujeto en situación y 
de la situación. ``El sujeto de la educación es un sujeto fundamentalmente colectivo porque surge de una combinación 
de distintos elementos, sin los cuales no sería posible ( maestros, estudiantes, conocimientos, prácticas.) Por lo 
tanto, no hay un sujeto preexistente, sino que hay un sujeto de y en la situaciones educativas." \citep{cerletti2008}  La constitución 
subjetiva no es previsible, está librada al azar del encuentro, al que no preexiste. Ello implica tomar en serio el 
caracter productivo del sistema escolar, poniendo en el centro de los procesos de constitución subjetiva la historia 
del dispositivo escolar y la naturaleza del proyecto escolar, y entendiendo la escolarización como parte del diseño del 
desarrollo humano historicamente producido, por lo tanto contingente.

\section {El sujeto como problema.}

La cuestión del sujeto está atravesada por los debates teóricos y políticos que sucita la crisis de la noción
moderna de sujeto. En el mundo metateórico de la modernidad, el sujeto consituía el núcleo duro de una identidad que 
se reconocía a si misma como tal, que se diferenciaba del objeto que tenía enfrente y prescindía de cualquier otredad. Esta 
noción de sujeto moderno está firmemente apoyada sobre el \textit{cogito cartesiano:} un sujeto ``unitario, autocentrado,
racionall, consciente de sí mismo" \citep{pineau1999}
El discurso de la modernidad ha ofrecido distintos modos la imagen de un ser consciente, racional y libre,
cuya razón lo llevaría a controlar el mundo natural, disminuyendo la incertidumbre \citep{beck2006}

La mirada de la pedagogía sobre los alumnos es tributaria de esa noción moderna de sujeto. Señala Pianeau que 
la modernidad convierte a la educación en un proceso por el cual el hombre se vuelve hombre, sujeto moderno,
cartesiano, con las características antes apuntadas \citep{pineau1999}. La educación tenía ademas la expectativa de eliminar
las limitaciones derivadas del nacimiento y de permitir que los humanos tuvieran su lugar en el tejido social sobre 
la base de su libre decisión y su actuación individual, para las cuales los formaría la escuela.
Sin embargo, el sujeto de la educación así definido no ha estado excento de tensiones. Así, no puede dejar de 
señalarse la paradoja de la eduación concebida por la tradición iluminista: por un lado, debería satisfacer el objetivo
 de autonomía del sujeto, haciéndolo capaz de servirse de su propia razón sin tutelas ajenas; pero al mismo tiempo debía 
 responder a la necesidad social de que el sujeto fuera gobernable\citep{cerletti2008}.
 Somos testigos, sujetos y objetos, de una fenomenal fractura del discurso de la modernidad en torno al sujeto. Los 
 planteos de los pensadores coinciden en generalizar la caída del proyecto ilustrado y el agotamiento de la razón moderna. Sin 
 necesidad de adscribir plenamenta a esa conincidencia, podemos reconocer que transitamos un momento histórico atravesado por 
 las ideas de lo provisorio, fugaz y transitorio, y también por la noción de riesgo debido a la socialización de las destrucciones
  de la naturaleza \citep{beck2006}. Con relación a estos procesos de rupturas aceleradas, se ubica una crisis de sentido del 
  discurso emancipador del humanismo moderno. La confusión, la desorientación, el desconcierto, se expresan en la desconfianza hacia 
  las posibilidades emancipadoras de la sociedad. Butler acuña el concepto de \textit{vidas precarias} para denunciar la diferencia 
  entre el desamparo original, propio de la condición humana, y el desamparo restrictivo de lo humano que se basa en la exclusión: 
  ``la vida se cuida y se mantiene diferencialmente y existen formas radicalmente diferentes de distribución de la vulnerabilidad
  física del hombre a lo largo del planeta. Ciertas vidas estan altamente protegidas, y el atentado contra su santidad basta para movilizar
  las fuerzas de la guerra. Otras vidas no gozan de una apoyo tan inmediato y furioso, y no se calificarán 
  incluso como `vidas que valgan la pena'" \citep{butler2006}. La crítica a la exclusión se extiende al mundo escolar: es suficientemente 
  conocida la sospecha sobre la capacidad de la educación escolarizada para producir igualdad en quienes la sociedad discrimina y excluye.
  En virtud de lo expuesto, tratar la problematica del sujeto de la educación en la formación de maestros y profesores implica proponer un 
  debate sobre las formas de producción de subjetividades, pero también sobre el sentido y la crisis de la educación moderna. La 
  escolarización es \textit{una manera} de dar tratamiento a la humanidad; una manera en ocasiones violeta, ya que implica 
  obligaciones de asistencia, permanencia, trabajo, logros, para evitar sanciones,así sea la del fracaso \citep{baqueroyterigi1996}.
  Si bien esta caracterización puede parecer excesivamente dura con respecto a una práctica que se considera no sólo deseable sino un 
  derecho subjetivo, nos permite insistir en que la cuestión del sujeto de la educación y de la educación debe ser tratada sin que 
  nuestro compromiso con los \textit{derechos del niño} nos lleve a desconocer el carácter \textit{político} del proyecto escolar y 
  las decisiones que involucra sobre la manera en que nuestras sociedades (atravesadas por la escuela) producen subjetivación.

\section{Las diferentes disciplinas que convergen en el estudio de la problematica del sujeto}

Podemos definir el sujeto de/en la educación como un \textit{sistema complejo}. Esto significa que estamos frente a un objeto caracterizado 
por fenómenos que están determinados por procesos donde entran en acción elementos que pertenecen al dominio de distintas
 disciplinas \citep{garcia1986}. Aceptado que el campo del sujeto es un campo en construcción y también sumamente dinámico, es necesario 
 sopesar la dificultad que impone, a una unidad curricular centrada en el sujeto de/en la educación, la necesaria concurrencia de las disciplinas 
 muy diversas. Entre ellas, la filosofía que aporta debates sobre la noción misma del sujeto y sobre la crisis de la modernidad; la sociología 
 , que aproxima nociones importantes para plantear las dinámicas de la acción de los sujetos; la antropología, que ilumina contrastivamente aspectos 
 poco visibles de la experiencia contemporánea; la psicología que, en la medida en que puede desprenderse de constitución subjetiva; y, por supuesto, 
 la pedagogía. Sin que este listado sea eshaustivo, es suficientemente amplio como para que quede planteada una aproximación a la complejidad del 
 problema teórico que se tiene por delante.

 Ahora bien, durante décadas las disciplinas modernas han desarrollado especializándose, bajo el supuesto de que cada una podía abordar dimensiones 
 diferentes de la experiencia humana. Problemas teóricos importantes como la oposición entre objetivismo y subjetivismo en la producción 
 de la acción humana resultan en parte efectos de esos modos \textit{disciplinados} de acumulación de saber.
 La oposición objetivismo/subjetivismo ilusta las dificultades teóricas que podemos encontrar para la articulación de contenido de distintas 
 disciplinas en torno a la problematica del sujeto, sobre todo si no queremos reducir el tratamiento de esta problematica a una simple yuxtaposición 
 de saberes disciplinados.

 \section {El sujeto de/en la eduación. Sujetos y alumnos}

 El otro análisis \citep{baqueroyterigi1996} hemos mostrado las características de las escuelas y del dispositivo escolar en su conjunto como 
 peculiares de un cierto proyecto político sobre la infancia. Si la escuela es un producto histórico, sus susjetos también lo son. En nuestra 
 contidianeidad se nos confunden las características de los chicos de seis años o de los de adoslescentes de trece con características de los 
 alumnos de primer grado de la escuela primaria o de los de primer año de nuestra tradicional escuela secundaria. Sin embargo, un niño y un 
 alumno, un adolescente y un estudiante, no son equivalentes.

 Un niño, un adolescente, no es un ente natural: es un sujeto producido socialmente en unas determinadas condiciones sociohistóricas. En 
 sociedades cada vez más diferenciadas, como la nuestra la homogeneidad del universo familiar y de los procesos de crianza que se presupone 
 el concepto naturalizado de infancia está visiblemente desmentida. No se puede pensar en saberes sobre el desarrollo subjetivo despojados 
 de referencias a las prácticas instituídas en que se constituyen los sujetos; por ejemplo, no es posible estudiar de modo \textit{neutral} 
 las adquisiciones de conocmientos, como si la crianza o la vida cotidiana de los niños fueran contextos \textit{naturales} \citep{castorina2007}, 
 para luego ver como funcionan esos conocimientos en la escuela.
 Por otro lado la escuela somete a los niños y adolescentes a un cierto régimen de trabajo (que creemos justificado por el beneficio de los 
 aprendizajes) que interviene productivamente en un desarrollo subjetivo que no puede pretenderse ``natural". El régimen de trabajo escolar 
 contribuye a constituirlos no sólo como alumnos sino también como niños adolescentes. La 
 escuela define lo que es esperable y lo que no: no sólo lo que significa un ``alumno medio", sino 
 también un `niño normal", ofreciendo a ambos como términos de comparación para el 
 comprotamiento y el desempeño de los sujetos reales.
 Este material se contruye desde una perspectiva advertida acerca del modo en que la escuela 
 genera la condición de \textit{alumno} y sobre la influencia del dispositivo escolar en nuestra visión no 
 sólo \textit{naturalizada} sino tambien \textit{normalizada} de la infancia y de los procesos de constitución 
 subjetiva \citep{baqueroyterigi1996}. La naturalización dota al sujeto de rasgos y capacidades que 
 se pretenden intríncecos a su naturaleza infantil (o adolescente, o adulta); por su parte, la 
 normalización lo pretende transitando por un único camino de desarrollo según niveles 
 uniformes, y mide como desvío, como retraso, etc., todo alejamiento de ese camino. Una 
 asignatura sobre los sujetos de/en la educación pone en cuestión los discursos naturalistas y 
 normalizadores, sobre el desarrollo, constituídos en estrecha vinculación con las prácticas 
 instituidas \citep{walkerdine1995}, entre ellas las escolares. El reconocimiento de que la 
 producción de saberes sobre los sujetos depende de relaciones de fuerza y de prácticas que subyacen a ella 
 no impide sostener la producción de discuros según reglas y métodos de investigación, pero 
 exige incorporar a esta producción la crítica de las condiciones de producción de saberes y a sus efectos,
 como la naturalización de la vida psicológica.

 \bibliographystyle{apalike}
\bibliography{biblio}
\end {document}